\def\sv{0.6.2}


\title{Online versions of the Scsh documentation}

\subsubsection*{The documents}

\nocite{scsh-paper,scsh-man}

\bibliographystyle{plain}
\bibliography{../bigbib}

\subsubsection*{Generating them}

\enumerate
\item  Get and unpack 
\urlh{http://www.scsh.net}{Scsh}.  
This will create a directory \p{scsh-|sv} 
(modulo version number).  

\item  Get and install 
\urlh{../tex2page-doc.html}{TeX2page}.  This will create a 
directory \p{tex2page}.   

(TeX2page is known to install on Scsh on Unix.  
However, for the least installation hassle,
consider
running TeX2page on
\urlh{http://www.plt-scheme.org/software/mzscheme}{MzScheme}.
TeX2page runs ``out of the box'' on MzScheme on any
operating system.)

\item  Put a copy or link of
\p{tex2page/t2p-example/scsh-paper.t2p} in
\p{scsh-|sv/doc/scsh-paper}.  Alternatively, ensure that 
\p{scsh-paper.t2p} is in one of the directories
mentioned in the environment variable
\p{TIIPINPUTS}.  (The \p{TIIPINPUTS} value looks like
the \p{PATH} value, ie, it is a list of directories,
colon-separated in Unix and semicolon-separated in
Windows.)

\item  Put a copy or link of \p{tex2page/t2p-example/man.t2p} 
in \p{scsh-|sv/doc/scsh-manual}.  Alternatively,
ensure that \p{man.t2p} is in \p{TIIPINPUTS}.

\item  In each of the directories \p{scsh-|sv/doc}
and \p{scsh-|sv/doc/scsh-manual}, create a file \p{.tex2page.hdir}
that contains the word \p{html}.  This tells TeX2page to
create the HTML files in a subdirectory \p{html}.  (This
step is optional, but helps to avoid cluttering
the source directory.)

\item  Cd to \p{scsh-|sv/doc/scsh-paper} and run \p{tex2page scsh-paper}.
(Do it twice to resolve cross-references.)

\item  Cd to \p{scsh-|sv/doc/scsh-manual} and run \p{tex2page man}.
(Do it twice.)

\item  Browse \p{scsh-|sv/doc/scsh-paper/html/scsh-paper.html} and 
\p{scsh-|sv/doc/scsh-manual/html/man.html} using your 
favorite HTML browser.  
\endenumerate

\bye
