\input texrc
\input btxmac
\input texnames.sty
\let\TZPtexlayout=1
\activettchar`

\newcount\qno

\def\question{\medbreak
  \global\advance\qno by 1
\noindent
  {\bf \the\qno.}\enspace
\bgroup
\aftergroup\egroup
\aftergroup\smallskip
\it}

%\title{TeX2page \\~\\ Frequently Asked Questions}
\def\TZPtitle{TeX2page FAQ}
\title{\TeX2page \\ Frequently Asked Questions}

\medskip

{\question Do I need to know Scheme or Common Lisp in order to be able
to use \TeX2page?}

No.  Even for those occasions where you need to
modify \TeX2page’s behavior, basic \TeX\ macros are
usually sufficient.  However, if you wish to extend
\TeX2page in peculiar and/or idiosyncratic ways,
familiarity with Scheme or Common Lisp will help.

{\question But I do need Scheme or Common Lisp to run \TeX2page?}

You need either a Scheme dialect or Common Lisp.

{\question This sounds like too much work!  Look, I am on Linux or Cygwin.
Shouldn’t I be able to install \TeX2page like other GNU software and have it
“just work”,
instead of worrying about Scheme, Common Lisp and their myriad
dialects?}

Well, you do need to have a Scheme or Common Lisp installed on your system.
Type `./configure --help` for guidance.

{\question I need a converter for my \LaTeX\ documents.
Doesn’t \TeX2page convert only plain \TeX?}

\TeX2page converts \LaTeX\ as well as plain \TeX\ documents.

{\question Do I need to make changes to my document to make it
suitable for \TeX2page?}

Typically, this isn’t necessary.  If \TeX2page
fails to convert your legacy document, you can put
\TeX2page-specific information in a `.t2p` file, without
modifying the legacy document.  See the
\urlh{index.html}{manual}.

{\question Why does my document have broken images?}

\TeX2page makes images using a combination of \TeX,
Dvips, Ghostscript and the NetPBM library.  Make sure
you have running versions of these programs on your
system.  They are all freely available from the
Web.

{\question Couldn’t the images be a bit larger?}

Use the `\imgpreamble` command to specify a higher
magnification for your images.

{\question How do I make my HTML file be a little
less drab?}

You need style sheets.  The \TeX2page manual tells you
how to incorporate them.

{\question Is there a way to avoid having my converted
document mention \TeX2page?}

Read about the `\TZPcolophoncredit` flag in the
\TeX2page manual.

{\question Do I really have to read the \TeX2page
manual?}

I suppose you could consult it only when you run into
problems, but do consider reading it.  It is a fine
manual.

{\question Why can’t I just use \LaTeX2HTML~\cite{latex2html}?}

You certainly can.  Converting TeX source to other
formats inevitably involves compromises, so different
converters will have different strengths, different
approaches to extensibility, and
appeal to different tastes.  You should pick the one
that best fits your needs.

{\question What are some of the other converters?}

HeVeA~\cite{hevea}, \TeX4ht~\cite{tex4ht}, TtH~\cite{tth}.

{\question I like getting both a
printable and an online document from the same source.
But hasn’t Texinfo~\cite{texinfo} already solved that problem?}

Actually the problem \TeX2page attempts to solve is getting
high-quality online texts {\em from
arbitrary \TeX\
source} (which we  already know to produce high-quality
printed output).  Texinfo requires its source to be in a
restricted Texinfo format.

{\question OK.  I am ready to give \TeX2page a try.
Where do I get it?}

The official \TeX2page website is
\urlh{index.html}{\path{http://ds26gte.github.io/tex2page/index.html}}.
The download link is just below the title.  Appendix D
contains installation instructions.

\beginsection Works cited

\bibliographystyle{plain}
\bibliography{tex2page}

\bye
